
% Default to the notebook output style

    


% Inherit from the specified cell style.




    
\documentclass[11pt]{article}

    
    
    \usepackage[T1]{fontenc}
    % Nicer default font (+ math font) than Computer Modern for most use cases
    \usepackage{mathpazo}

    % Basic figure setup, for now with no caption control since it's done
    % automatically by Pandoc (which extracts ![](path) syntax from Markdown).
    \usepackage{graphicx}
    % We will generate all images so they have a width \maxwidth. This means
    % that they will get their normal width if they fit onto the page, but
    % are scaled down if they would overflow the margins.
    \makeatletter
    \def\maxwidth{\ifdim\Gin@nat@width>\linewidth\linewidth
    \else\Gin@nat@width\fi}
    \makeatother
    \let\Oldincludegraphics\includegraphics
    % Set max figure width to be 80% of text width, for now hardcoded.
    \renewcommand{\includegraphics}[1]{\Oldincludegraphics[width=.8\maxwidth]{#1}}
    % Ensure that by default, figures have no caption (until we provide a
    % proper Figure object with a Caption API and a way to capture that
    % in the conversion process - todo).
    \usepackage{caption}
    \DeclareCaptionLabelFormat{nolabel}{}
    \captionsetup{labelformat=nolabel}

    \usepackage{adjustbox} % Used to constrain images to a maximum size 
    \usepackage{xcolor} % Allow colors to be defined
    \usepackage{enumerate} % Needed for markdown enumerations to work
    \usepackage{geometry} % Used to adjust the document margins
    \usepackage{amsmath} % Equations
    \usepackage{amssymb} % Equations
    \usepackage{textcomp} % defines textquotesingle
    % Hack from http://tex.stackexchange.com/a/47451/13684:
    \AtBeginDocument{%
        \def\PYZsq{\textquotesingle}% Upright quotes in Pygmentized code
    }
    \usepackage{upquote} % Upright quotes for verbatim code
    \usepackage{eurosym} % defines \euro
    \usepackage[mathletters]{ucs} % Extended unicode (utf-8) support
    \usepackage[utf8x]{inputenc} % Allow utf-8 characters in the tex document
    \usepackage{fancyvrb} % verbatim replacement that allows latex
    \usepackage{grffile} % extends the file name processing of package graphics 
                         % to support a larger range 
    % The hyperref package gives us a pdf with properly built
    % internal navigation ('pdf bookmarks' for the table of contents,
    % internal cross-reference links, web links for URLs, etc.)
    \usepackage{hyperref}
    \usepackage{longtable} % longtable support required by pandoc >1.10
    \usepackage{booktabs}  % table support for pandoc > 1.12.2
    \usepackage[inline]{enumitem} % IRkernel/repr support (it uses the enumerate* environment)
    \usepackage[normalem]{ulem} % ulem is needed to support strikethroughs (\sout)
                                % normalem makes italics be italics, not underlines
    

    
    
    % Colors for the hyperref package
    \definecolor{urlcolor}{rgb}{0,.145,.698}
    \definecolor{linkcolor}{rgb}{.71,0.21,0.01}
    \definecolor{citecolor}{rgb}{.12,.54,.11}

    % ANSI colors
    \definecolor{ansi-black}{HTML}{3E424D}
    \definecolor{ansi-black-intense}{HTML}{282C36}
    \definecolor{ansi-red}{HTML}{E75C58}
    \definecolor{ansi-red-intense}{HTML}{B22B31}
    \definecolor{ansi-green}{HTML}{00A250}
    \definecolor{ansi-green-intense}{HTML}{007427}
    \definecolor{ansi-yellow}{HTML}{DDB62B}
    \definecolor{ansi-yellow-intense}{HTML}{B27D12}
    \definecolor{ansi-blue}{HTML}{208FFB}
    \definecolor{ansi-blue-intense}{HTML}{0065CA}
    \definecolor{ansi-magenta}{HTML}{D160C4}
    \definecolor{ansi-magenta-intense}{HTML}{A03196}
    \definecolor{ansi-cyan}{HTML}{60C6C8}
    \definecolor{ansi-cyan-intense}{HTML}{258F8F}
    \definecolor{ansi-white}{HTML}{C5C1B4}
    \definecolor{ansi-white-intense}{HTML}{A1A6B2}

    % commands and environments needed by pandoc snippets
    % extracted from the output of `pandoc -s`
    \providecommand{\tightlist}{%
      \setlength{\itemsep}{0pt}\setlength{\parskip}{0pt}}
    \DefineVerbatimEnvironment{Highlighting}{Verbatim}{commandchars=\\\{\}}
    % Add ',fontsize=\small' for more characters per line
    \newenvironment{Shaded}{}{}
    \newcommand{\KeywordTok}[1]{\textcolor[rgb]{0.00,0.44,0.13}{\textbf{{#1}}}}
    \newcommand{\DataTypeTok}[1]{\textcolor[rgb]{0.56,0.13,0.00}{{#1}}}
    \newcommand{\DecValTok}[1]{\textcolor[rgb]{0.25,0.63,0.44}{{#1}}}
    \newcommand{\BaseNTok}[1]{\textcolor[rgb]{0.25,0.63,0.44}{{#1}}}
    \newcommand{\FloatTok}[1]{\textcolor[rgb]{0.25,0.63,0.44}{{#1}}}
    \newcommand{\CharTok}[1]{\textcolor[rgb]{0.25,0.44,0.63}{{#1}}}
    \newcommand{\StringTok}[1]{\textcolor[rgb]{0.25,0.44,0.63}{{#1}}}
    \newcommand{\CommentTok}[1]{\textcolor[rgb]{0.38,0.63,0.69}{\textit{{#1}}}}
    \newcommand{\OtherTok}[1]{\textcolor[rgb]{0.00,0.44,0.13}{{#1}}}
    \newcommand{\AlertTok}[1]{\textcolor[rgb]{1.00,0.00,0.00}{\textbf{{#1}}}}
    \newcommand{\FunctionTok}[1]{\textcolor[rgb]{0.02,0.16,0.49}{{#1}}}
    \newcommand{\RegionMarkerTok}[1]{{#1}}
    \newcommand{\ErrorTok}[1]{\textcolor[rgb]{1.00,0.00,0.00}{\textbf{{#1}}}}
    \newcommand{\NormalTok}[1]{{#1}}
    
    % Additional commands for more recent versions of Pandoc
    \newcommand{\ConstantTok}[1]{\textcolor[rgb]{0.53,0.00,0.00}{{#1}}}
    \newcommand{\SpecialCharTok}[1]{\textcolor[rgb]{0.25,0.44,0.63}{{#1}}}
    \newcommand{\VerbatimStringTok}[1]{\textcolor[rgb]{0.25,0.44,0.63}{{#1}}}
    \newcommand{\SpecialStringTok}[1]{\textcolor[rgb]{0.73,0.40,0.53}{{#1}}}
    \newcommand{\ImportTok}[1]{{#1}}
    \newcommand{\DocumentationTok}[1]{\textcolor[rgb]{0.73,0.13,0.13}{\textit{{#1}}}}
    \newcommand{\AnnotationTok}[1]{\textcolor[rgb]{0.38,0.63,0.69}{\textbf{\textit{{#1}}}}}
    \newcommand{\CommentVarTok}[1]{\textcolor[rgb]{0.38,0.63,0.69}{\textbf{\textit{{#1}}}}}
    \newcommand{\VariableTok}[1]{\textcolor[rgb]{0.10,0.09,0.49}{{#1}}}
    \newcommand{\ControlFlowTok}[1]{\textcolor[rgb]{0.00,0.44,0.13}{\textbf{{#1}}}}
    \newcommand{\OperatorTok}[1]{\textcolor[rgb]{0.40,0.40,0.40}{{#1}}}
    \newcommand{\BuiltInTok}[1]{{#1}}
    \newcommand{\ExtensionTok}[1]{{#1}}
    \newcommand{\PreprocessorTok}[1]{\textcolor[rgb]{0.74,0.48,0.00}{{#1}}}
    \newcommand{\AttributeTok}[1]{\textcolor[rgb]{0.49,0.56,0.16}{{#1}}}
    \newcommand{\InformationTok}[1]{\textcolor[rgb]{0.38,0.63,0.69}{\textbf{\textit{{#1}}}}}
    \newcommand{\WarningTok}[1]{\textcolor[rgb]{0.38,0.63,0.69}{\textbf{\textit{{#1}}}}}
    
    
    % Define a nice break command that doesn't care if a line doesn't already
    % exist.
    \def\br{\hspace*{\fill} \\* }
    % Math Jax compatability definitions
    \def\gt{>}
    \def\lt{<}
    % Document parameters
    \title{karthik.krishnaswamy17@gmail.com\_op1}
    
    
    

    % Pygments definitions
    
\makeatletter
\def\PY@reset{\let\PY@it=\relax \let\PY@bf=\relax%
    \let\PY@ul=\relax \let\PY@tc=\relax%
    \let\PY@bc=\relax \let\PY@ff=\relax}
\def\PY@tok#1{\csname PY@tok@#1\endcsname}
\def\PY@toks#1+{\ifx\relax#1\empty\else%
    \PY@tok{#1}\expandafter\PY@toks\fi}
\def\PY@do#1{\PY@bc{\PY@tc{\PY@ul{%
    \PY@it{\PY@bf{\PY@ff{#1}}}}}}}
\def\PY#1#2{\PY@reset\PY@toks#1+\relax+\PY@do{#2}}

\expandafter\def\csname PY@tok@w\endcsname{\def\PY@tc##1{\textcolor[rgb]{0.73,0.73,0.73}{##1}}}
\expandafter\def\csname PY@tok@c\endcsname{\let\PY@it=\textit\def\PY@tc##1{\textcolor[rgb]{0.25,0.50,0.50}{##1}}}
\expandafter\def\csname PY@tok@cp\endcsname{\def\PY@tc##1{\textcolor[rgb]{0.74,0.48,0.00}{##1}}}
\expandafter\def\csname PY@tok@k\endcsname{\let\PY@bf=\textbf\def\PY@tc##1{\textcolor[rgb]{0.00,0.50,0.00}{##1}}}
\expandafter\def\csname PY@tok@kp\endcsname{\def\PY@tc##1{\textcolor[rgb]{0.00,0.50,0.00}{##1}}}
\expandafter\def\csname PY@tok@kt\endcsname{\def\PY@tc##1{\textcolor[rgb]{0.69,0.00,0.25}{##1}}}
\expandafter\def\csname PY@tok@o\endcsname{\def\PY@tc##1{\textcolor[rgb]{0.40,0.40,0.40}{##1}}}
\expandafter\def\csname PY@tok@ow\endcsname{\let\PY@bf=\textbf\def\PY@tc##1{\textcolor[rgb]{0.67,0.13,1.00}{##1}}}
\expandafter\def\csname PY@tok@nb\endcsname{\def\PY@tc##1{\textcolor[rgb]{0.00,0.50,0.00}{##1}}}
\expandafter\def\csname PY@tok@nf\endcsname{\def\PY@tc##1{\textcolor[rgb]{0.00,0.00,1.00}{##1}}}
\expandafter\def\csname PY@tok@nc\endcsname{\let\PY@bf=\textbf\def\PY@tc##1{\textcolor[rgb]{0.00,0.00,1.00}{##1}}}
\expandafter\def\csname PY@tok@nn\endcsname{\let\PY@bf=\textbf\def\PY@tc##1{\textcolor[rgb]{0.00,0.00,1.00}{##1}}}
\expandafter\def\csname PY@tok@ne\endcsname{\let\PY@bf=\textbf\def\PY@tc##1{\textcolor[rgb]{0.82,0.25,0.23}{##1}}}
\expandafter\def\csname PY@tok@nv\endcsname{\def\PY@tc##1{\textcolor[rgb]{0.10,0.09,0.49}{##1}}}
\expandafter\def\csname PY@tok@no\endcsname{\def\PY@tc##1{\textcolor[rgb]{0.53,0.00,0.00}{##1}}}
\expandafter\def\csname PY@tok@nl\endcsname{\def\PY@tc##1{\textcolor[rgb]{0.63,0.63,0.00}{##1}}}
\expandafter\def\csname PY@tok@ni\endcsname{\let\PY@bf=\textbf\def\PY@tc##1{\textcolor[rgb]{0.60,0.60,0.60}{##1}}}
\expandafter\def\csname PY@tok@na\endcsname{\def\PY@tc##1{\textcolor[rgb]{0.49,0.56,0.16}{##1}}}
\expandafter\def\csname PY@tok@nt\endcsname{\let\PY@bf=\textbf\def\PY@tc##1{\textcolor[rgb]{0.00,0.50,0.00}{##1}}}
\expandafter\def\csname PY@tok@nd\endcsname{\def\PY@tc##1{\textcolor[rgb]{0.67,0.13,1.00}{##1}}}
\expandafter\def\csname PY@tok@s\endcsname{\def\PY@tc##1{\textcolor[rgb]{0.73,0.13,0.13}{##1}}}
\expandafter\def\csname PY@tok@sd\endcsname{\let\PY@it=\textit\def\PY@tc##1{\textcolor[rgb]{0.73,0.13,0.13}{##1}}}
\expandafter\def\csname PY@tok@si\endcsname{\let\PY@bf=\textbf\def\PY@tc##1{\textcolor[rgb]{0.73,0.40,0.53}{##1}}}
\expandafter\def\csname PY@tok@se\endcsname{\let\PY@bf=\textbf\def\PY@tc##1{\textcolor[rgb]{0.73,0.40,0.13}{##1}}}
\expandafter\def\csname PY@tok@sr\endcsname{\def\PY@tc##1{\textcolor[rgb]{0.73,0.40,0.53}{##1}}}
\expandafter\def\csname PY@tok@ss\endcsname{\def\PY@tc##1{\textcolor[rgb]{0.10,0.09,0.49}{##1}}}
\expandafter\def\csname PY@tok@sx\endcsname{\def\PY@tc##1{\textcolor[rgb]{0.00,0.50,0.00}{##1}}}
\expandafter\def\csname PY@tok@m\endcsname{\def\PY@tc##1{\textcolor[rgb]{0.40,0.40,0.40}{##1}}}
\expandafter\def\csname PY@tok@gh\endcsname{\let\PY@bf=\textbf\def\PY@tc##1{\textcolor[rgb]{0.00,0.00,0.50}{##1}}}
\expandafter\def\csname PY@tok@gu\endcsname{\let\PY@bf=\textbf\def\PY@tc##1{\textcolor[rgb]{0.50,0.00,0.50}{##1}}}
\expandafter\def\csname PY@tok@gd\endcsname{\def\PY@tc##1{\textcolor[rgb]{0.63,0.00,0.00}{##1}}}
\expandafter\def\csname PY@tok@gi\endcsname{\def\PY@tc##1{\textcolor[rgb]{0.00,0.63,0.00}{##1}}}
\expandafter\def\csname PY@tok@gr\endcsname{\def\PY@tc##1{\textcolor[rgb]{1.00,0.00,0.00}{##1}}}
\expandafter\def\csname PY@tok@ge\endcsname{\let\PY@it=\textit}
\expandafter\def\csname PY@tok@gs\endcsname{\let\PY@bf=\textbf}
\expandafter\def\csname PY@tok@gp\endcsname{\let\PY@bf=\textbf\def\PY@tc##1{\textcolor[rgb]{0.00,0.00,0.50}{##1}}}
\expandafter\def\csname PY@tok@go\endcsname{\def\PY@tc##1{\textcolor[rgb]{0.53,0.53,0.53}{##1}}}
\expandafter\def\csname PY@tok@gt\endcsname{\def\PY@tc##1{\textcolor[rgb]{0.00,0.27,0.87}{##1}}}
\expandafter\def\csname PY@tok@err\endcsname{\def\PY@bc##1{\setlength{\fboxsep}{0pt}\fcolorbox[rgb]{1.00,0.00,0.00}{1,1,1}{\strut ##1}}}
\expandafter\def\csname PY@tok@kc\endcsname{\let\PY@bf=\textbf\def\PY@tc##1{\textcolor[rgb]{0.00,0.50,0.00}{##1}}}
\expandafter\def\csname PY@tok@kd\endcsname{\let\PY@bf=\textbf\def\PY@tc##1{\textcolor[rgb]{0.00,0.50,0.00}{##1}}}
\expandafter\def\csname PY@tok@kn\endcsname{\let\PY@bf=\textbf\def\PY@tc##1{\textcolor[rgb]{0.00,0.50,0.00}{##1}}}
\expandafter\def\csname PY@tok@kr\endcsname{\let\PY@bf=\textbf\def\PY@tc##1{\textcolor[rgb]{0.00,0.50,0.00}{##1}}}
\expandafter\def\csname PY@tok@bp\endcsname{\def\PY@tc##1{\textcolor[rgb]{0.00,0.50,0.00}{##1}}}
\expandafter\def\csname PY@tok@fm\endcsname{\def\PY@tc##1{\textcolor[rgb]{0.00,0.00,1.00}{##1}}}
\expandafter\def\csname PY@tok@vc\endcsname{\def\PY@tc##1{\textcolor[rgb]{0.10,0.09,0.49}{##1}}}
\expandafter\def\csname PY@tok@vg\endcsname{\def\PY@tc##1{\textcolor[rgb]{0.10,0.09,0.49}{##1}}}
\expandafter\def\csname PY@tok@vi\endcsname{\def\PY@tc##1{\textcolor[rgb]{0.10,0.09,0.49}{##1}}}
\expandafter\def\csname PY@tok@vm\endcsname{\def\PY@tc##1{\textcolor[rgb]{0.10,0.09,0.49}{##1}}}
\expandafter\def\csname PY@tok@sa\endcsname{\def\PY@tc##1{\textcolor[rgb]{0.73,0.13,0.13}{##1}}}
\expandafter\def\csname PY@tok@sb\endcsname{\def\PY@tc##1{\textcolor[rgb]{0.73,0.13,0.13}{##1}}}
\expandafter\def\csname PY@tok@sc\endcsname{\def\PY@tc##1{\textcolor[rgb]{0.73,0.13,0.13}{##1}}}
\expandafter\def\csname PY@tok@dl\endcsname{\def\PY@tc##1{\textcolor[rgb]{0.73,0.13,0.13}{##1}}}
\expandafter\def\csname PY@tok@s2\endcsname{\def\PY@tc##1{\textcolor[rgb]{0.73,0.13,0.13}{##1}}}
\expandafter\def\csname PY@tok@sh\endcsname{\def\PY@tc##1{\textcolor[rgb]{0.73,0.13,0.13}{##1}}}
\expandafter\def\csname PY@tok@s1\endcsname{\def\PY@tc##1{\textcolor[rgb]{0.73,0.13,0.13}{##1}}}
\expandafter\def\csname PY@tok@mb\endcsname{\def\PY@tc##1{\textcolor[rgb]{0.40,0.40,0.40}{##1}}}
\expandafter\def\csname PY@tok@mf\endcsname{\def\PY@tc##1{\textcolor[rgb]{0.40,0.40,0.40}{##1}}}
\expandafter\def\csname PY@tok@mh\endcsname{\def\PY@tc##1{\textcolor[rgb]{0.40,0.40,0.40}{##1}}}
\expandafter\def\csname PY@tok@mi\endcsname{\def\PY@tc##1{\textcolor[rgb]{0.40,0.40,0.40}{##1}}}
\expandafter\def\csname PY@tok@il\endcsname{\def\PY@tc##1{\textcolor[rgb]{0.40,0.40,0.40}{##1}}}
\expandafter\def\csname PY@tok@mo\endcsname{\def\PY@tc##1{\textcolor[rgb]{0.40,0.40,0.40}{##1}}}
\expandafter\def\csname PY@tok@ch\endcsname{\let\PY@it=\textit\def\PY@tc##1{\textcolor[rgb]{0.25,0.50,0.50}{##1}}}
\expandafter\def\csname PY@tok@cm\endcsname{\let\PY@it=\textit\def\PY@tc##1{\textcolor[rgb]{0.25,0.50,0.50}{##1}}}
\expandafter\def\csname PY@tok@cpf\endcsname{\let\PY@it=\textit\def\PY@tc##1{\textcolor[rgb]{0.25,0.50,0.50}{##1}}}
\expandafter\def\csname PY@tok@c1\endcsname{\let\PY@it=\textit\def\PY@tc##1{\textcolor[rgb]{0.25,0.50,0.50}{##1}}}
\expandafter\def\csname PY@tok@cs\endcsname{\let\PY@it=\textit\def\PY@tc##1{\textcolor[rgb]{0.25,0.50,0.50}{##1}}}

\def\PYZbs{\char`\\}
\def\PYZus{\char`\_}
\def\PYZob{\char`\{}
\def\PYZcb{\char`\}}
\def\PYZca{\char`\^}
\def\PYZam{\char`\&}
\def\PYZlt{\char`\<}
\def\PYZgt{\char`\>}
\def\PYZsh{\char`\#}
\def\PYZpc{\char`\%}
\def\PYZdl{\char`\$}
\def\PYZhy{\char`\-}
\def\PYZsq{\char`\'}
\def\PYZdq{\char`\"}
\def\PYZti{\char`\~}
% for compatibility with earlier versions
\def\PYZat{@}
\def\PYZlb{[}
\def\PYZrb{]}
\makeatother


    % Exact colors from NB
    \definecolor{incolor}{rgb}{0.0, 0.0, 0.5}
    \definecolor{outcolor}{rgb}{0.545, 0.0, 0.0}



    
    % Prevent overflowing lines due to hard-to-break entities
    \sloppy 
    % Setup hyperref package
    \hypersetup{
      breaklinks=true,  % so long urls are correctly broken across lines
      colorlinks=true,
      urlcolor=urlcolor,
      linkcolor=linkcolor,
      citecolor=citecolor,
      }
    % Slightly bigger margins than the latex defaults
    
    \geometry{verbose,tmargin=1in,bmargin=1in,lmargin=1in,rmargin=1in}
    
    

    \begin{document}
    
    
    \maketitle
    
    

    
    \begin{Verbatim}[commandchars=\\\{\}]
{\color{incolor}In [{\color{incolor} }]:} \PY{c+c1}{\PYZsh{}1) Write a function that inputs a number and prints the multiplication table of that number}
\end{Verbatim}


    \begin{Verbatim}[commandchars=\\\{\}]
{\color{incolor}In [{\color{incolor}1}]:} \PY{k}{def} \PY{n+nf}{mult}\PY{p}{(}\PY{n}{n}\PY{p}{)}\PY{p}{:}
            \PY{l+s+sd}{\PYZdq{}\PYZdq{}\PYZdq{}}
        \PY{l+s+sd}{        Function that prints the multiplication table of that number.}
        \PY{l+s+sd}{    \PYZdq{}\PYZdq{}\PYZdq{}}
            \PY{k}{for} \PY{n}{i} \PY{o+ow}{in} \PY{n+nb}{range} \PY{p}{(}\PY{l+m+mi}{1}\PY{p}{,}\PY{l+m+mi}{11}\PY{p}{)}\PY{p}{:}
                \PY{n+nb}{print}\PY{p}{(}\PY{l+s+s2}{\PYZdq{}}\PY{l+s+si}{\PYZob{}0\PYZcb{}}\PY{l+s+s2}{ * }\PY{l+s+si}{\PYZob{}1\PYZcb{}}\PY{l+s+s2}{ =}\PY{l+s+si}{\PYZob{}2\PYZcb{}}\PY{l+s+s2}{\PYZdq{}}\PY{o}{.}\PY{n}{format}\PY{p}{(}\PY{n}{n}\PY{p}{,}\PY{n}{i}\PY{p}{,}\PY{n}{n}\PY{o}{*}\PY{n}{i}\PY{p}{)}\PY{p}{)}
        
        \PY{n}{n}\PY{o}{=}\PY{n+nb}{int}\PY{p}{(}\PY{n+nb}{input}\PY{p}{(}\PY{l+s+s2}{\PYZdq{}}\PY{l+s+s2}{Enter the number:}\PY{l+s+s2}{\PYZdq{}}\PY{p}{)}\PY{p}{)}
        \PY{n}{mult}\PY{p}{(}\PY{n}{n}\PY{p}{)}
\end{Verbatim}


    \begin{Verbatim}[commandchars=\\\{\}]
Enter the number:5
5 * 1 =5
5 * 2 =10
5 * 3 =15
5 * 4 =20
5 * 5 =25
5 * 6 =30
5 * 7 =35
5 * 8 =40
5 * 9 =45
5 * 10 =50

    \end{Verbatim}

    \begin{Verbatim}[commandchars=\\\{\}]
{\color{incolor}In [{\color{incolor}2}]:} \PY{c+c1}{\PYZsh{}2)Write a program to print twin primes less than 1000. }
        \PY{c+c1}{\PYZsh{}If two consecutive odd numbers are both prime then they are known as twin primes.}
\end{Verbatim}


    \begin{Verbatim}[commandchars=\\\{\}]
{\color{incolor}In [{\color{incolor}3}]:} \PY{k+kn}{import} \PY{n+nn}{numpy} \PY{k}{as} \PY{n+nn}{np}
        \PY{k}{def} \PY{n+nf}{prime}\PY{p}{(}\PY{n}{n}\PY{p}{)}\PY{p}{:}
            \PY{l+s+sd}{\PYZdq{}\PYZdq{}\PYZdq{}}
        \PY{l+s+sd}{    Find Prime CF of two numbers using (sieve of eratosthenes Alogrithm)}
        \PY{l+s+sd}{    \PYZdq{}\PYZdq{}\PYZdq{}}
            \PY{n}{p}\PY{o}{=}\PY{n}{np}\PY{o}{.}\PY{n}{arange}\PY{p}{(}\PY{n}{n}\PY{p}{)}
            \PY{n}{p}\PY{p}{[}\PY{l+m+mi}{0}\PY{p}{]}\PY{o}{=}\PY{l+m+mi}{0}
            \PY{n}{p}\PY{p}{[}\PY{l+m+mi}{1}\PY{p}{]}\PY{o}{=}\PY{l+m+mi}{0}
            \PY{k}{for} \PY{n}{i} \PY{o+ow}{in} \PY{n+nb}{range} \PY{p}{(}\PY{l+m+mi}{2}\PY{p}{,}\PY{n}{n}\PY{p}{)}\PY{p}{:}
                \PY{n}{p}\PY{p}{[}\PY{n}{i}\PY{p}{]}\PY{o}{=}\PY{l+m+mi}{1}
        
            \PY{k}{for} \PY{n}{i} \PY{o+ow}{in} \PY{n+nb}{range} \PY{p}{(}\PY{l+m+mi}{2}\PY{p}{,}\PY{n}{n}\PY{p}{)}\PY{p}{:}
                \PY{k}{if} \PY{n}{p}\PY{p}{[}\PY{n}{i}\PY{p}{]}\PY{o}{==}\PY{l+m+mi}{1}\PY{p}{:}
                    \PY{n}{j}\PY{o}{=}\PY{l+m+mi}{2}                            
                \PY{k}{while}\PY{p}{(}\PY{n}{i}\PY{o}{*}\PY{n}{j}\PY{o}{\PYZlt{}}\PY{n}{n}\PY{p}{)}\PY{p}{:}
                    \PY{n}{p}\PY{p}{[}\PY{n}{i}\PY{o}{*}\PY{n}{j}\PY{p}{]}\PY{o}{=}\PY{l+m+mi}{0}
                    \PY{n}{j}\PY{o}{+}\PY{o}{=}\PY{l+m+mi}{1}
        
            \PY{n+nb}{print}\PY{p}{(}\PY{l+s+s2}{\PYZdq{}}\PY{l+s+s2}{Twin prime numbers till }\PY{l+s+si}{\PYZob{}\PYZcb{}}\PY{l+s+s2}{\PYZdq{}}\PY{o}{.}\PY{n}{format}\PY{p}{(}\PY{n}{n}\PY{p}{)}\PY{p}{)}
            \PY{k}{for} \PY{n}{i} \PY{o+ow}{in} \PY{n+nb}{range} \PY{p}{(}\PY{n}{n}\PY{o}{\PYZhy{}}\PY{l+m+mi}{1}\PY{p}{)}\PY{p}{:}
                \PY{k}{if} \PY{n}{p}\PY{p}{[}\PY{n}{i}\PY{p}{]}\PY{o}{==}\PY{l+m+mi}{1} \PY{o+ow}{and} \PY{n}{p}\PY{p}{[}\PY{n}{i}\PY{o}{+}\PY{l+m+mi}{2}\PY{p}{]}\PY{o}{==}\PY{l+m+mi}{1}\PY{p}{:}
                    \PY{n+nb}{print}\PY{p}{(}\PY{n}{i}\PY{p}{,}\PY{n}{i}\PY{o}{+}\PY{l+m+mi}{2}\PY{p}{)}
        
        \PY{n}{prime}\PY{p}{(}\PY{l+m+mi}{1000}\PY{p}{)}
                
\end{Verbatim}


    \begin{Verbatim}[commandchars=\\\{\}]
Twin prime numbers till 1000
3 5
5 7
11 13
17 19
29 31
41 43
59 61
71 73
101 103
107 109
137 139
149 151
179 181
191 193
197 199
227 229
239 241
269 271
281 283
311 313
347 349
419 421
431 433
461 463
521 523
569 571
599 601
617 619
641 643
659 661
809 811
821 823
827 829
857 859
881 883

    \end{Verbatim}

    \begin{Verbatim}[commandchars=\\\{\}]
{\color{incolor}In [{\color{incolor} }]:} \PY{c+c1}{\PYZsh{}3) Write a program to find out the prime factors of a number. Example: prime factors of 56 \PYZhy{}}
        \PY{c+c1}{\PYZsh{}2, 2, 2,7)}
\end{Verbatim}


    \begin{Verbatim}[commandchars=\\\{\}]
{\color{incolor}In [{\color{incolor}7}]:} \PY{k+kn}{import} \PY{n+nn}{math} 
        \PY{k}{def} \PY{n+nf}{primefactors}\PY{p}{(}\PY{n}{n}\PY{p}{)}\PY{p}{:}
            \PY{n}{c}\PY{o}{=}\PY{l+m+mi}{0}
            \PY{k}{for} \PY{n}{i} \PY{o+ow}{in} \PY{n+nb}{range} \PY{p}{(}\PY{l+m+mi}{2}\PY{p}{,}\PY{n+nb}{int}\PY{p}{(}\PY{n}{math}\PY{o}{.}\PY{n}{sqrt}\PY{p}{(}\PY{n}{n}\PY{p}{)}\PY{p}{)}\PY{o}{+}\PY{l+m+mi}{1}\PY{p}{)}\PY{p}{:}
                    \PY{k}{while} \PY{p}{(}\PY{n}{n}\PY{o}{\PYZpc{}}\PY{k}{i}==0):
                        \PY{n}{n}\PY{o}{=}\PY{n}{n}\PY{o}{/}\PY{n}{i}
                        \PY{n}{c}\PY{o}{+}\PY{o}{=}\PY{l+m+mi}{1}
                        \PY{n+nb}{print}\PY{p}{(}\PY{n}{i}\PY{p}{)}
            \PY{k}{if} \PY{n}{c}\PY{o}{==}\PY{l+m+mi}{0} \PY{p}{:}
                \PY{n+nb}{print}\PY{p}{(}\PY{l+s+s2}{\PYZdq{}}\PY{l+s+si}{\PYZob{}\PYZcb{}}\PY{l+s+s2}{ is a prime number.No prime factors for prime number.}\PY{l+s+s2}{\PYZdq{}}\PY{o}{.}\PY{n}{format}\PY{p}{(}\PY{n}{n}\PY{p}{)}\PY{p}{)}
        
        \PY{n}{i}\PY{o}{=}\PY{n+nb}{int}\PY{p}{(}\PY{n+nb}{input}\PY{p}{(}\PY{l+s+s2}{\PYZdq{}}\PY{l+s+s2}{Enter any number to find prime factors:}\PY{l+s+s2}{\PYZdq{}}\PY{p}{)}\PY{p}{)}
        \PY{n}{primefactors}\PY{p}{(}\PY{n}{i}\PY{p}{)}
\end{Verbatim}


    \begin{Verbatim}[commandchars=\\\{\}]
Enter any number to find prime factors:15
3

    \end{Verbatim}

    \begin{Verbatim}[commandchars=\\\{\}]
{\color{incolor}In [{\color{incolor}8}]:} \PY{c+c1}{\PYZsh{}4) Write a program to implement these formulae of permutations and combinations.}
        \PY{c+c1}{\PYZsh{}Number of permutations of n objects taken r at a time: p(n, r) = n! / (n\PYZhy{}r)!. Number of}
        \PY{c+c1}{\PYZsh{}combinations of n objects taken r at a time is: c(n, r) = n! / (r!*(n\PYZhy{}r)!) = p(n,r) / r!}
\end{Verbatim}


    \begin{Verbatim}[commandchars=\\\{\}]
{\color{incolor}In [{\color{incolor}9}]:} \PY{k}{def} \PY{n+nf}{fact}\PY{p}{(}\PY{n}{num}\PY{p}{)}\PY{p}{:}
            \PY{l+s+sd}{\PYZdq{}\PYZdq{}\PYZdq{}}
        \PY{l+s+sd}{    Old School method to find factorial of a number without using recursive.}
        \PY{l+s+sd}{    \PYZdq{}\PYZdq{}\PYZdq{}}
            \PY{k}{if} \PY{n}{num}\PY{o}{==}\PY{l+m+mi}{0} \PY{o+ow}{or} \PY{n}{num}\PY{o}{==}\PY{l+m+mi}{1}\PY{p}{:}
                \PY{k}{return} \PY{l+m+mi}{1}
            \PY{k}{else}\PY{p}{:}
                \PY{n}{p}\PY{o}{=}\PY{l+m+mi}{1}
                \PY{k}{for} \PY{n}{i} \PY{o+ow}{in} \PY{n+nb}{range} \PY{p}{(}\PY{l+m+mi}{2}\PY{p}{,}\PY{n}{num}\PY{o}{+}\PY{l+m+mi}{1}\PY{p}{)}\PY{p}{:}
                    \PY{n}{p}\PY{o}{*}\PY{o}{=}\PY{n}{i}
                \PY{k}{return} \PY{n}{p}    
                    
        \PY{k}{def} \PY{n+nf}{permutation}\PY{p}{(}\PY{n}{n}\PY{p}{,}\PY{n}{r}\PY{p}{)}\PY{p}{:}
            \PY{n}{result}\PY{o}{=}\PY{n}{fact}\PY{p}{(}\PY{n}{n}\PY{p}{)}\PY{o}{/}\PY{n}{fact}\PY{p}{(}\PY{n}{n}\PY{o}{\PYZhy{}}\PY{n}{r}\PY{p}{)}
            \PY{n+nb}{print}\PY{p}{(}\PY{l+s+s2}{\PYZdq{}}\PY{l+s+s2}{Permutation of p(}\PY{l+s+si}{\PYZob{}0\PYZcb{}}\PY{l+s+s2}{,}\PY{l+s+si}{\PYZob{}1\PYZcb{}}\PY{l+s+s2}{)=}\PY{l+s+si}{\PYZob{}2\PYZcb{}}\PY{l+s+s2}{\PYZdq{}}\PY{o}{.}\PY{n}{format}\PY{p}{(}\PY{n}{n}\PY{p}{,}\PY{n}{r}\PY{p}{,}\PY{n}{result}\PY{p}{)}\PY{p}{)}
        
        \PY{k}{def} \PY{n+nf}{combination}\PY{p}{(}\PY{n}{n}\PY{p}{,}\PY{n}{r}\PY{p}{)}\PY{p}{:}
            \PY{n}{result}\PY{o}{=}\PY{n}{fact}\PY{p}{(}\PY{n}{n}\PY{p}{)}\PY{o}{/}\PY{p}{(}\PY{n}{fact}\PY{p}{(}\PY{n}{r}\PY{p}{)}\PY{o}{*}\PY{n}{fact}\PY{p}{(}\PY{n}{n}\PY{o}{\PYZhy{}}\PY{n}{r}\PY{p}{)}\PY{p}{)}
            \PY{n+nb}{print}\PY{p}{(}\PY{l+s+s2}{\PYZdq{}}\PY{l+s+s2}{Combination of p(}\PY{l+s+si}{\PYZob{}0\PYZcb{}}\PY{l+s+s2}{,}\PY{l+s+si}{\PYZob{}1\PYZcb{}}\PY{l+s+s2}{)=}\PY{l+s+si}{\PYZob{}2\PYZcb{}}\PY{l+s+s2}{\PYZdq{}}\PY{o}{.}\PY{n}{format}\PY{p}{(}\PY{n}{n}\PY{p}{,}\PY{n}{r}\PY{p}{,}\PY{n}{result}\PY{p}{)}\PY{p}{)}
        
        \PY{n}{n}\PY{o}{=}\PY{n+nb}{int}\PY{p}{(}\PY{n+nb}{input}\PY{p}{(}\PY{l+s+s2}{\PYZdq{}}\PY{l+s+s2}{Enter number of Objects:}\PY{l+s+s2}{\PYZdq{}}\PY{p}{)}\PY{p}{)}
        \PY{n}{r}\PY{o}{=}\PY{n+nb}{int}\PY{p}{(}\PY{n+nb}{input}\PY{p}{(}\PY{l+s+s2}{\PYZdq{}}\PY{l+s+s2}{Enter the time of objects taken at time: }\PY{l+s+s2}{\PYZdq{}}\PY{p}{)}\PY{p}{)}
        \PY{n}{combination}\PY{p}{(}\PY{n}{n}\PY{p}{,}\PY{n}{r}\PY{p}{)}
        \PY{n}{permutation}\PY{p}{(}\PY{n}{n}\PY{p}{,}\PY{n}{r}\PY{p}{)}
\end{Verbatim}


    \begin{Verbatim}[commandchars=\\\{\}]
Enter number of Objects:6
Enter the time of objects taken at time: 3
Combination of p(6,3)=20.0
Permutation of p(6,3)=120.0

    \end{Verbatim}

    \begin{Verbatim}[commandchars=\\\{\}]
{\color{incolor}In [{\color{incolor} }]:} \PY{c+c1}{\PYZsh{}5)Write a function that converts a decimal number to binary number}
\end{Verbatim}


    \begin{Verbatim}[commandchars=\\\{\}]
{\color{incolor}In [{\color{incolor}12}]:} \PY{k+kn}{import} \PY{n+nn}{numpy} \PY{k}{as} \PY{n+nn}{np}
         \PY{k}{def} \PY{n+nf}{dec2bin}\PY{p}{(}\PY{n}{n}\PY{p}{)}\PY{p}{:}
             \PY{n+nb}{bin}\PY{o}{=}\PY{p}{[}\PY{p}{]}
             \PY{k}{while} \PY{p}{(}\PY{n}{n}\PY{o}{\PYZgt{}}\PY{l+m+mi}{0}\PY{p}{)}\PY{p}{:}
                 \PY{n}{r}\PY{o}{=}\PY{n+nb}{int}\PY{p}{(}\PY{n}{n}\PY{o}{\PYZpc{}}\PY{k}{2})
                 \PY{n}{n}\PY{o}{=}\PY{n+nb}{int}\PY{p}{(}\PY{n}{n}\PY{o}{/}\PY{l+m+mi}{2}\PY{p}{)}
                 \PY{n+nb}{bin}\PY{o}{.}\PY{n}{append}\PY{p}{(}\PY{n}{r}\PY{p}{)}
             \PY{n+nb}{bin}\PY{o}{.}\PY{n}{reverse}\PY{p}{(}\PY{p}{)}
             
             \PY{c+c1}{\PYZsh{} Adding the numbers by using power of 10\PYZsq{}s using one\PYZsq{}s twos\PYZsq{}....}
             \PY{n}{res} \PY{o}{=} \PY{n+nb}{sum}\PY{p}{(}\PY{n}{d} \PY{o}{*} \PY{l+m+mi}{10}\PY{o}{*}\PY{o}{*}\PY{n}{i} \PY{k}{for} \PY{n}{i}\PY{p}{,} \PY{n}{d} \PY{o+ow}{in} \PY{n+nb}{enumerate}\PY{p}{(}\PY{n+nb}{bin}\PY{p}{[}\PY{p}{:}\PY{p}{:}\PY{o}{\PYZhy{}}\PY{l+m+mi}{1}\PY{p}{]}\PY{p}{)}\PY{p}{)} 
             \PY{k}{return} \PY{n}{res}
         \PY{n}{n}\PY{o}{=}\PY{n+nb}{int}\PY{p}{(}\PY{n+nb}{input}\PY{p}{(}\PY{l+s+s2}{\PYZdq{}}\PY{l+s+s2}{Enter number in decimal format:}\PY{l+s+s2}{\PYZdq{}}\PY{p}{)}\PY{p}{)}
         \PY{n+nb}{print}\PY{p}{(}\PY{l+s+s2}{\PYZdq{}}\PY{l+s+si}{\PYZob{}\PYZcb{}}\PY{l+s+s2}{ in binary:}\PY{l+s+si}{\PYZob{}\PYZcb{}}\PY{l+s+s2}{\PYZdq{}}\PY{o}{.}\PY{n}{format}\PY{p}{(}\PY{n}{n}\PY{p}{,}\PY{n}{dec2bin}\PY{p}{(}\PY{n}{n}\PY{p}{)}\PY{p}{)}\PY{p}{)}
\end{Verbatim}


    \begin{Verbatim}[commandchars=\\\{\}]
Enter number in decimal format:112
112 in binary:1110000

    \end{Verbatim}

    \begin{Verbatim}[commandchars=\\\{\}]
{\color{incolor}In [{\color{incolor} }]:} \PY{c+c1}{\PYZsh{}6)Write a function cubesum() that accepts an integer and returns the sum of the cubes of}
        \PY{c+c1}{\PYZsh{}individual digits of that number. Use this function to make functions PrintArmstrong() and}
        \PY{c+c1}{\PYZsh{}isArmstrong() to print Armstrong numbers and to find whether is an Armstrong number.}
\end{Verbatim}


    \begin{Verbatim}[commandchars=\\\{\}]
{\color{incolor}In [{\color{incolor}13}]:} \PY{k}{def} \PY{n+nf}{isArmstrong}\PY{p}{(}\PY{n}{s}\PY{p}{,}\PY{n}{n}\PY{p}{)}\PY{p}{:}
             \PY{k}{if} \PY{n}{s}\PY{o}{==}\PY{n}{n}\PY{p}{:}
                 \PY{n+nb}{print}\PY{p}{(}\PY{l+s+s2}{\PYZdq{}}\PY{l+s+si}{\PYZob{}\PYZcb{}}\PY{l+s+s2}{ is Armstrong number}\PY{l+s+s2}{\PYZdq{}}\PY{o}{.}\PY{n}{format}\PY{p}{(}\PY{n}{n}\PY{p}{)}\PY{p}{)}
             \PY{k}{else}\PY{p}{:}
                 \PY{n+nb}{print}\PY{p}{(}\PY{l+s+s2}{\PYZdq{}}\PY{l+s+si}{\PYZob{}\PYZcb{}}\PY{l+s+s2}{ is not a Armstrong number}\PY{l+s+s2}{\PYZdq{}}\PY{o}{.}\PY{n}{format}\PY{p}{(}\PY{n}{n}\PY{p}{)}\PY{p}{)}
         
         \PY{k}{def} \PY{n+nf}{cubesum}\PY{p}{(}\PY{n}{n}\PY{p}{)}\PY{p}{:}
             \PY{n}{l}\PY{o}{=}\PY{p}{[}\PY{n+nb}{int}\PY{p}{(}\PY{n}{x}\PY{p}{)} \PY{k}{for} \PY{n}{x} \PY{o+ow}{in} \PY{n+nb}{str}\PY{p}{(}\PY{n}{n}\PY{p}{)}\PY{p}{]}
             \PY{n}{s}\PY{o}{=}\PY{l+m+mi}{0}
             \PY{k}{for} \PY{n}{ele} \PY{o+ow}{in} \PY{n}{l}\PY{p}{:}
                 \PY{n}{s}\PY{o}{+}\PY{o}{=}\PY{n}{ele}\PY{o}{*}\PY{o}{*}\PY{l+m+mi}{3}
             \PY{n}{isArmstrong}\PY{p}{(}\PY{n}{s}\PY{p}{,}\PY{n}{n}\PY{p}{)}
         
         \PY{k}{def} \PY{n+nf}{PrintArmstrong}\PY{p}{(}\PY{n}{n}\PY{p}{)}\PY{p}{:}
             \PY{n}{cubesum}\PY{p}{(}\PY{n}{n}\PY{p}{)}    
          
         \PY{n}{n}\PY{o}{=}\PY{n+nb}{int}\PY{p}{(}\PY{n+nb}{input}\PY{p}{(}\PY{l+s+s2}{\PYZdq{}}\PY{l+s+s2}{Enter the number:}\PY{l+s+s2}{\PYZdq{}}\PY{p}{)}\PY{p}{)}
         \PY{n}{PrintArmstrong}\PY{p}{(}\PY{n}{n}\PY{p}{)}
\end{Verbatim}


    \begin{Verbatim}[commandchars=\\\{\}]
Enter the number:407
407 is Armstrong number

    \end{Verbatim}

    \begin{Verbatim}[commandchars=\\\{\}]
{\color{incolor}In [{\color{incolor} }]:} \PY{c+c1}{\PYZsh{}Write a function prodDigits() that inputs a number and returns the product of digits of that number.}
\end{Verbatim}


    \begin{Verbatim}[commandchars=\\\{\}]
{\color{incolor}In [{\color{incolor}14}]:} \PY{k}{def} \PY{n+nf}{prodDigits}\PY{p}{(}\PY{n}{n}\PY{p}{)}\PY{p}{:}
             \PY{n}{l}\PY{o}{=}\PY{p}{[}\PY{n+nb}{int}\PY{p}{(}\PY{n}{x}\PY{p}{)} \PY{k}{for} \PY{n}{x} \PY{o+ow}{in} \PY{n+nb}{str}\PY{p}{(}\PY{n}{n}\PY{p}{)}\PY{p}{]}
             \PY{n}{p}\PY{o}{=}\PY{l+m+mi}{1}
             \PY{k}{for} \PY{n}{ele} \PY{o+ow}{in} \PY{n}{l}\PY{p}{:}
                 \PY{n}{p}\PY{o}{*}\PY{o}{=}\PY{n}{ele}
             \PY{k}{return} \PY{n}{p}    
         
         \PY{k}{if} \PY{n+nv+vm}{\PYZus{}\PYZus{}name\PYZus{}\PYZus{}} \PY{o}{==} \PY{l+s+s1}{\PYZsq{}}\PY{l+s+s1}{\PYZus{}\PYZus{}main\PYZus{}\PYZus{}}\PY{l+s+s1}{\PYZsq{}}\PY{p}{:}
             \PY{n}{n}\PY{o}{=}\PY{n+nb}{int}\PY{p}{(}\PY{n+nb}{input}\PY{p}{(}\PY{l+s+s2}{\PYZdq{}}\PY{l+s+s2}{Enter the number to find products of digits:}\PY{l+s+s2}{\PYZdq{}}\PY{p}{)}\PY{p}{)}
             \PY{n+nb}{print}\PY{p}{(}\PY{l+s+s2}{\PYZdq{}}\PY{l+s+si}{\PYZob{}\PYZcb{}}\PY{l+s+s2}{ is the product of each digits in }\PY{l+s+si}{\PYZob{}\PYZcb{}}\PY{l+s+s2}{\PYZdq{}}\PY{o}{.}\PY{n}{format}\PY{p}{(}\PY{n}{prodDigits}\PY{p}{(}\PY{n}{n}\PY{p}{)}\PY{p}{,}\PY{n}{n}\PY{p}{)}\PY{p}{)}
\end{Verbatim}


    \begin{Verbatim}[commandchars=\\\{\}]
Enter the number to find products of digits:143
12 is the product of each digits in 143

    \end{Verbatim}

    \begin{Verbatim}[commandchars=\\\{\}]
{\color{incolor}In [{\color{incolor} }]:} \PY{c+c1}{\PYZsh{}8 Find digital Multiplicative digital root of n.}
\end{Verbatim}


    \begin{Verbatim}[commandchars=\\\{\}]
{\color{incolor}In [{\color{incolor}29}]:} \PY{c+c1}{\PYZsh{}import prodDigits}
         \PY{n}{n}\PY{o}{=}\PY{n+nb}{int}\PY{p}{(}\PY{n+nb}{input}\PY{p}{(}\PY{l+s+s2}{\PYZdq{}}\PY{l+s+s2}{Enter the number:}\PY{l+s+s2}{\PYZdq{}}\PY{p}{)}\PY{p}{)}
         \PY{n}{c}\PY{o}{=}\PY{l+m+mi}{0}
         \PY{k}{if} \PY{n}{n}\PY{o}{\PYZlt{}}\PY{l+m+mi}{9}\PY{p}{:}
             \PY{n+nb}{print}\PY{p}{(}\PY{l+s+s2}{\PYZdq{}}\PY{l+s+s2}{MDR:}\PY{l+s+si}{\PYZob{}\PYZcb{}}\PY{l+s+s2}{ MPersistence:}\PY{l+s+si}{\PYZob{}\PYZcb{}}\PY{l+s+s2}{\PYZdq{}}\PY{o}{.}\PY{n}{format}\PY{p}{(}\PY{n}{n}\PY{p}{,}\PY{n}{c}\PY{p}{)}\PY{p}{)}
         \PY{k}{else}\PY{p}{:}
             \PY{k}{while}\PY{p}{(}\PY{n}{n}\PY{o}{\PYZgt{}}\PY{l+m+mi}{9}\PY{p}{)}\PY{p}{:}
                 \PY{n}{n}\PY{o}{=}\PY{n}{prodDigits}\PY{p}{(}\PY{n}{n}\PY{p}{)}
                 \PY{k}{if} \PY{n}{n}\PY{o}{\PYZgt{}}\PY{o}{=}\PY{l+m+mi}{0}\PY{p}{:}
                     \PY{n}{c}\PY{o}{+}\PY{o}{=}\PY{l+m+mi}{1}
             \PY{n+nb}{print}\PY{p}{(}\PY{l+s+s2}{\PYZdq{}}\PY{l+s+s2}{MDR:}\PY{l+s+si}{\PYZob{}\PYZcb{}}\PY{l+s+s2}{ MPersistence:}\PY{l+s+si}{\PYZob{}\PYZcb{}}\PY{l+s+s2}{\PYZdq{}}\PY{o}{.}\PY{n}{format}\PY{p}{(}\PY{n}{n}\PY{p}{,}\PY{n}{c}\PY{p}{)}\PY{p}{)}    
\end{Verbatim}


    \begin{Verbatim}[commandchars=\\\{\}]
Enter the number:341
MDR:2 MPersistence:2

    \end{Verbatim}

    \begin{Verbatim}[commandchars=\\\{\}]
{\color{incolor}In [{\color{incolor} }]:} \PY{c+c1}{\PYZsh{}9 Write a function sumPdivisors() that finds the sum of proper divisors of a number.}
\end{Verbatim}


    \begin{Verbatim}[commandchars=\\\{\}]
{\color{incolor}In [{\color{incolor}30}]:} \PY{k}{def} \PY{n+nf}{sumPdivisors}\PY{p}{(}\PY{n}{n}\PY{p}{)}\PY{p}{:}
             \PY{n}{l}\PY{o}{=}\PY{p}{[}\PY{p}{]}
             \PY{n}{s}\PY{o}{=}\PY{l+m+mi}{0}
             \PY{k}{if} \PY{n}{n}\PY{o}{==}\PY{l+m+mi}{1}\PY{p}{:}
                 \PY{k}{return} \PY{p}{[}\PY{p}{[}\PY{l+m+mi}{1}\PY{p}{]}\PY{p}{,}\PY{l+m+mi}{1}\PY{p}{]}
             \PY{k}{else}\PY{p}{:}    
                 \PY{k}{for} \PY{n}{i} \PY{o+ow}{in} \PY{n+nb}{range} \PY{p}{(}\PY{l+m+mi}{1}\PY{p}{,}\PY{n+nb}{int}\PY{p}{(}\PY{n}{n}\PY{o}{/}\PY{l+m+mi}{2}\PY{p}{)}\PY{o}{+}\PY{l+m+mi}{1}\PY{p}{)}\PY{p}{:}
                     \PY{k}{if} \PY{p}{(}\PY{n}{n}\PY{o}{\PYZpc{}}\PY{k}{i}==0):
                         \PY{n}{l}\PY{o}{.}\PY{n}{append}\PY{p}{(}\PY{n}{i}\PY{p}{)}
                         \PY{n}{s}\PY{o}{+}\PY{o}{=}\PY{n}{i}
                 \PY{k}{return} \PY{p}{[}\PY{n}{l}\PY{p}{,}\PY{n}{s}\PY{p}{]}
                  
         \PY{k}{if} \PY{n+nv+vm}{\PYZus{}\PYZus{}name\PYZus{}\PYZus{}} \PY{o}{==} \PY{l+s+s2}{\PYZdq{}}\PY{l+s+s2}{\PYZus{}\PYZus{}main\PYZus{}\PYZus{}}\PY{l+s+s2}{\PYZdq{}}\PY{p}{:}
             \PY{n}{n}\PY{o}{=}\PY{n+nb}{int}\PY{p}{(}\PY{n+nb}{input}\PY{p}{(}\PY{l+s+s2}{\PYZdq{}}\PY{l+s+s2}{Enter the number:}\PY{l+s+s2}{\PYZdq{}}\PY{p}{)}\PY{p}{)}
             \PY{n}{o}\PY{o}{=}\PY{n}{sumPdivisors}\PY{p}{(}\PY{n}{n}\PY{p}{)}
             \PY{n+nb}{print}\PY{p}{(}\PY{l+s+s2}{\PYZdq{}}\PY{l+s+s2}{Divisor of }\PY{l+s+si}{\PYZob{}\PYZcb{}}\PY{l+s+s2}{ :}\PY{l+s+si}{\PYZob{}\PYZcb{}}\PY{l+s+se}{\PYZbs{}n}\PY{l+s+s2}{Sum of these Divisor:}\PY{l+s+si}{\PYZob{}\PYZcb{}}\PY{l+s+s2}{\PYZdq{}}\PY{o}{.}\PY{n}{format}\PY{p}{(}\PY{n}{n}\PY{p}{,}\PY{n}{o}\PY{p}{[}\PY{l+m+mi}{0}\PY{p}{]}\PY{p}{,}\PY{n}{o}\PY{p}{[}\PY{l+m+mi}{1}\PY{p}{]}\PY{p}{)}\PY{p}{)}
\end{Verbatim}


    \begin{Verbatim}[commandchars=\\\{\}]
Enter the number:18
Divisor of 18 :[1, 2, 3, 6, 9]
Sum of these Divisor:21

    \end{Verbatim}

    \begin{Verbatim}[commandchars=\\\{\}]
{\color{incolor}In [{\color{incolor} }]:} \PY{c+c1}{\PYZsh{}10 Write a program to print all the perfect numbers in a given range}
\end{Verbatim}


    \begin{Verbatim}[commandchars=\\\{\}]
{\color{incolor}In [{\color{incolor}35}]:} \PY{c+c1}{\PYZsh{}import sumPdivisors}
         
         \PY{k}{def} \PY{n+nf}{isPerfect}\PY{p}{(}\PY{n}{n}\PY{p}{)}\PY{p}{:}
             \PY{n}{r}\PY{o}{=}\PY{n}{sumPdivisors}\PY{p}{(}\PY{n}{n}\PY{p}{)}
             \PY{k}{if} \PY{n}{r}\PY{p}{[}\PY{l+m+mi}{1}\PY{p}{]}\PY{o}{==}\PY{n}{n} \PY{p}{:}
                 \PY{n+nb}{print}\PY{p}{(}\PY{l+s+s2}{\PYZdq{}}\PY{l+s+s2}{Yeah! }\PY{l+s+si}{\PYZob{}\PYZcb{}}\PY{l+s+s2}{ is  a perfect number}\PY{l+s+s2}{\PYZdq{}}\PY{o}{.}\PY{n}{format}\PY{p}{(}\PY{n}{n}\PY{p}{)}\PY{p}{)}
             \PY{k}{else}\PY{p}{:}
                 \PY{n+nb}{print}\PY{p}{(}\PY{l+s+s2}{\PYZdq{}}\PY{l+s+s2}{Sorry! }\PY{l+s+si}{\PYZob{}\PYZcb{}}\PY{l+s+s2}{ is  not a perfect number}\PY{l+s+s2}{\PYZdq{}}\PY{o}{.}\PY{n}{format}\PY{p}{(}\PY{n}{n}\PY{p}{)}\PY{p}{)}
         
         \PY{n}{n}\PY{o}{=}\PY{n+nb}{int}\PY{p}{(}\PY{n+nb}{input}\PY{p}{(}\PY{l+s+s2}{\PYZdq{}}\PY{l+s+s2}{Enter the number:}\PY{l+s+s2}{\PYZdq{}}\PY{p}{)}\PY{p}{)}
         \PY{n}{isPerfect}\PY{p}{(}\PY{n}{n}\PY{p}{)}
\end{Verbatim}


    \begin{Verbatim}[commandchars=\\\{\}]
Enter the number:8128
Yeah! 8128 is  a perfect number

    \end{Verbatim}

    \begin{Verbatim}[commandchars=\\\{\}]
{\color{incolor}In [{\color{incolor} }]:} \PY{c+c1}{\PYZsh{}11 Write a function to print pairs of amicable numbers in a range}
\end{Verbatim}


    \begin{Verbatim}[commandchars=\\\{\}]
{\color{incolor}In [{\color{incolor}37}]:} \PY{c+c1}{\PYZsh{}from code\PYZus{}9 import sumPdivisors}
         \PY{k}{def} \PY{n+nf}{checkAmicable}\PY{p}{(}\PY{n}{n}\PY{p}{)}\PY{p}{:}
             \PY{n}{a}\PY{o}{=}\PY{n+nb}{dict}\PY{p}{(}\PY{p}{)}
             \PY{n}{b}\PY{o}{=}\PY{n+nb}{dict}\PY{p}{(}\PY{p}{)}
             \PY{k}{for} \PY{n}{i} \PY{o+ow}{in} \PY{n+nb}{range} \PY{p}{(}\PY{n}{n}\PY{p}{)}\PY{p}{:}
                 \PY{n}{o}\PY{o}{=}\PY{n}{sumPdivisors}\PY{p}{(}\PY{n}{i}\PY{p}{)}\PY{p}{[}\PY{l+m+mi}{1}\PY{p}{]}
                 \PY{k}{if} \PY{n}{o}\PY{o}{\PYZgt{}}\PY{l+m+mi}{1} \PY{o+ow}{and} \PY{n}{o}\PY{o}{!=}\PY{n}{i}\PY{p}{:}
                     \PY{n}{a}\PY{o}{.}\PY{n}{update}\PY{p}{(}\PY{p}{\PYZob{}}\PY{n}{i}\PY{p}{:}\PY{n}{o}\PY{p}{\PYZcb{}}\PY{p}{)}
                     
             \PY{c+c1}{\PYZsh{} Old Fashion For Loop}
             \PY{c+c1}{\PYZsh{} for k in a.keys():}
             \PY{c+c1}{\PYZsh{}     for k1,v in a.items():}
             \PY{c+c1}{\PYZsh{}         if(k==v and a[v]==k1): }
             \PY{c+c1}{\PYZsh{}             print(k,a[k])}
         
             \PY{c+c1}{\PYZsh{} Comprehension list}
             \PY{n}{c}\PY{o}{=}\PY{p}{\PYZob{}}\PY{n}{k}\PY{p}{:}\PY{n}{a}\PY{p}{[}\PY{n}{k}\PY{p}{]} \PY{k}{for} \PY{n}{k} \PY{o+ow}{in} \PY{n}{a}\PY{o}{.}\PY{n}{keys}\PY{p}{(}\PY{p}{)} \PY{k}{for} \PY{n}{k1}\PY{p}{,}\PY{n}{v} \PY{o+ow}{in} \PY{n}{a}\PY{o}{.}\PY{n}{items}\PY{p}{(}\PY{p}{)} \PY{k}{if} \PY{p}{(}\PY{n}{k}\PY{o}{==}\PY{n}{v} \PY{o+ow}{and} \PY{n}{a}\PY{p}{[}\PY{n}{v}\PY{p}{]}\PY{o}{==}\PY{n}{k1}\PY{p}{)}\PY{p}{\PYZcb{}}
             \PY{n+nb}{print}\PY{p}{(}\PY{l+s+s2}{\PYZdq{}}\PY{l+s+s2}{Amicable pairs:}\PY{l+s+se}{\PYZbs{}n}\PY{l+s+si}{\PYZob{}\PYZcb{}}\PY{l+s+s2}{\PYZdq{}}\PY{o}{.}\PY{n}{format}\PY{p}{(}\PY{n}{c}\PY{p}{)}\PY{p}{)}
          
         \PY{n}{checkAmicable}\PY{p}{(}\PY{n+nb}{int}\PY{p}{(}\PY{n+nb}{input}\PY{p}{(}\PY{l+s+s2}{\PYZdq{}}\PY{l+s+s2}{Enter the range to find amicable numbers:}\PY{l+s+s2}{\PYZdq{}}\PY{p}{)}\PY{p}{)}\PY{p}{)}
\end{Verbatim}


    \begin{Verbatim}[commandchars=\\\{\}]
Enter the range to find amicable numbers:10000
Amicable pairs:
\{220: 284, 284: 220, 1184: 1210, 1210: 1184, 2620: 2924, 2924: 2620, 5020: 5564, 5564: 5020, 6232: 6368, 6368: 6232\}

    \end{Verbatim}

    \begin{Verbatim}[commandchars=\\\{\}]
{\color{incolor}In [{\color{incolor} }]:} \PY{c+c1}{\PYZsh{}12 Write a program which can filter odd numbers in a list by using filter function}
\end{Verbatim}


    \begin{Verbatim}[commandchars=\\\{\}]
{\color{incolor}In [{\color{incolor}38}]:} \PY{k}{def} \PY{n+nf}{oddNumbers}\PY{p}{(}\PY{n}{n}\PY{p}{)}\PY{p}{:}
             \PY{k}{if} \PY{n}{n}\PY{o}{\PYZpc{}}\PY{k}{2}!=0:
                 \PY{k}{return} \PY{k+kc}{True}
         
         \PY{n}{n}\PY{o}{=}\PY{n+nb}{int}\PY{p}{(}\PY{n+nb}{input}\PY{p}{(}\PY{l+s+s2}{\PYZdq{}}\PY{l+s+s2}{Enter the range:}\PY{l+s+s2}{\PYZdq{}}\PY{p}{)}\PY{p}{)}
         \PY{n}{l}\PY{o}{=}\PY{n+nb}{range}\PY{p}{(}\PY{n}{n}\PY{p}{)}
         \PY{n+nb}{print}\PY{p}{(}\PY{l+s+s2}{\PYZdq{}}\PY{l+s+s2}{Odd numbers under the range }\PY{l+s+si}{\PYZob{}\PYZcb{}}\PY{l+s+s2}{:}\PY{l+s+s2}{\PYZdq{}}\PY{o}{.}\PY{n}{format}\PY{p}{(}\PY{n}{n}\PY{p}{)}\PY{p}{)}
         \PY{n+nb}{print}\PY{p}{(}\PY{n+nb}{list}\PY{p}{(}\PY{n+nb}{filter}\PY{p}{(}\PY{n}{oddNumbers}\PY{p}{,}\PY{n}{l}\PY{p}{)}\PY{p}{)}\PY{p}{)}
\end{Verbatim}


    \begin{Verbatim}[commandchars=\\\{\}]
Enter the range:30
Odd numbers under the range 30:
[1, 3, 5, 7, 9, 11, 13, 15, 17, 19, 21, 23, 25, 27, 29]

    \end{Verbatim}

    \begin{Verbatim}[commandchars=\\\{\}]
{\color{incolor}In [{\color{incolor} }]:} \PY{c+c1}{\PYZsh{}13 Write a program which can map() to make a list whose elements are cube of elements in a given list}
\end{Verbatim}


    \begin{Verbatim}[commandchars=\\\{\}]
{\color{incolor}In [{\color{incolor}39}]:} \PY{k}{def} \PY{n+nf}{cubeElements}\PY{p}{(}\PY{n}{n}\PY{p}{)}\PY{p}{:}
             \PY{k}{return} \PY{n}{n}\PY{o}{*}\PY{o}{*}\PY{l+m+mi}{3}
         
         \PY{k}{if} \PY{n+nv+vm}{\PYZus{}\PYZus{}name\PYZus{}\PYZus{}} \PY{o}{==} \PY{l+s+s2}{\PYZdq{}}\PY{l+s+s2}{\PYZus{}\PYZus{}main\PYZus{}\PYZus{}}\PY{l+s+s2}{\PYZdq{}}\PY{p}{:}
             \PY{n}{n}\PY{o}{=}\PY{n+nb}{int}\PY{p}{(}\PY{n+nb}{input}\PY{p}{(}\PY{l+s+s2}{\PYZdq{}}\PY{l+s+s2}{Enter the range:}\PY{l+s+s2}{\PYZdq{}}\PY{p}{)}\PY{p}{)}
             \PY{n}{l}\PY{o}{=}\PY{n+nb}{range}\PY{p}{(}\PY{n}{n}\PY{p}{)}
             \PY{n}{o}\PY{o}{=}\PY{n+nb}{map}\PY{p}{(}\PY{n}{cubeElements}\PY{p}{,}\PY{n}{l}\PY{p}{)}
             \PY{n+nb}{print}\PY{p}{(}\PY{l+s+s2}{\PYZdq{}}\PY{l+s+s2}{List of cube elements:}\PY{l+s+se}{\PYZbs{}n}\PY{l+s+si}{\PYZob{}\PYZcb{}}\PY{l+s+s2}{\PYZdq{}}\PY{o}{.}\PY{n}{format}\PY{p}{(}\PY{n+nb}{list}\PY{p}{(}\PY{n}{o}\PY{p}{)}\PY{p}{)}\PY{p}{)}
\end{Verbatim}


    \begin{Verbatim}[commandchars=\\\{\}]
Enter the range:10
List of cube elements:
[0, 1, 8, 27, 64, 125, 216, 343, 512, 729]

    \end{Verbatim}

    \begin{Verbatim}[commandchars=\\\{\}]
{\color{incolor}In [{\color{incolor}40}]:} \PY{c+c1}{\PYZsh{}14 Write a program which can map() and filter() to make a list whose elements are cube of }
         \PY{c+c1}{\PYZsh{}even number in a given list}
\end{Verbatim}


    \begin{Verbatim}[commandchars=\\\{\}]
{\color{incolor}In [{\color{incolor}41}]:} \PY{c+c1}{\PYZsh{}from code\PYZus{}13 import cubeElements}
         
         \PY{k}{def} \PY{n+nf}{evenNumbers}\PY{p}{(}\PY{n}{n}\PY{p}{)}\PY{p}{:}
             \PY{k}{if} \PY{n}{n}\PY{o}{\PYZpc{}}\PY{k}{2}==0:
                 \PY{k}{return} \PY{k+kc}{True}
         
         \PY{n}{n}\PY{o}{=}\PY{n+nb}{int}\PY{p}{(}\PY{n+nb}{input}\PY{p}{(}\PY{l+s+s2}{\PYZdq{}}\PY{l+s+s2}{Enter the range:}\PY{l+s+s2}{\PYZdq{}}\PY{p}{)}\PY{p}{)}
         \PY{n}{l}\PY{o}{=}\PY{n+nb}{range}\PY{p}{(}\PY{l+m+mi}{1}\PY{p}{,}\PY{n}{n}\PY{p}{)}
         
         \PY{n}{evenList}\PY{o}{=}\PY{n+nb}{list}\PY{p}{(}\PY{n+nb}{filter}\PY{p}{(}\PY{n}{evenNumbers}\PY{p}{,}\PY{n}{l}\PY{p}{)}\PY{p}{)}
         \PY{n}{cube\PYZus{}of\PYZus{}even\PYZus{}list}\PY{o}{=}\PY{n+nb}{list}\PY{p}{(}\PY{n+nb}{map}\PY{p}{(}\PY{n}{cubeElements}\PY{p}{,}\PY{n}{evenList}\PY{p}{)}\PY{p}{)}
         
         \PY{n+nb}{print}\PY{p}{(}\PY{l+s+s2}{\PYZdq{}}\PY{l+s+s2}{Even number under }\PY{l+s+si}{\PYZob{}\PYZcb{}}\PY{l+s+s2}{:}\PY{l+s+se}{\PYZbs{}n}\PY{l+s+s2}{\PYZdq{}}\PY{o}{.}\PY{n}{format}\PY{p}{(}\PY{n}{evenList}\PY{p}{)}\PY{p}{)}
         \PY{n+nb}{print}\PY{p}{(}\PY{l+s+s2}{\PYZdq{}}\PY{l+s+s2}{Cube of even number list:}\PY{l+s+se}{\PYZbs{}n}\PY{l+s+si}{\PYZob{}\PYZcb{}}\PY{l+s+s2}{\PYZdq{}}\PY{o}{.}\PY{n}{format}\PY{p}{(}\PY{n}{cube\PYZus{}of\PYZus{}even\PYZus{}list}\PY{p}{)}\PY{p}{)}
\end{Verbatim}


    \begin{Verbatim}[commandchars=\\\{\}]
Enter the range:15
Even number under [2, 4, 6, 8, 10, 12, 14]:

Cube of even number list:
[8, 64, 216, 512, 1000, 1728, 2744]

    \end{Verbatim}


    % Add a bibliography block to the postdoc
    
    
    
    \end{document}
